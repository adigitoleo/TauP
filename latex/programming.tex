
\section{Programming Interface}

In addition to the command line interface, there are three ways to access
the toolkit from within other programs. The most straightforward is
through Java. Using Jacl provides a very nice way to write scripts that
use the tools without repeatedly starting up the Java virtual machine and
reloading models. Lastly, there is a C language interface, but it is
a bit less friendly. Descriptions of all three, with example programs
are below.

\subsection{Java}

The TauP package should be easily used by future Java programs. An
example is given illustrating the basics of using the package to
generate travel times.

First, instantiate a TauP\_Time object. This provides methods for
generating and
using travel times and should be sufficient for most purposes.
However, in order to actually generate anything useful, the TauP\_Time
object needs a TauModel. It can be loaded within the constructor for TauP\_Time
as a TauModel or with the model name.
It can changed later using either the \texttt{TauP\_Time.setTauModel(TauModel)} method of
TauP\_Time, or by passing the modelname to \texttt{TauP\_Time.loadTauModel(String)}.
The later is likely easier, and has the advantage of searching for the model in
the distribution jar file, the locations in the taup.model.path property,
and the current directory.

\texttt{TauP\_Time timeTool = new TauP\_Time("mymodel");}

In addition to the TauModel, a collection of phases is also needed. Again,
there are several ways of accomplishing this.
\texttt{parsePhaseList(String)} is likely the easiest method. A String is passed
with the phase names separated by commas and the phases are extracted and
appended. Phases can also be input more
directly with \texttt{setPhaseNames(String[])} ,
which sets the phases to be those in the array, and
\texttt{appendPhaseName(String)} which appends a phase to the list. Note that
these methods do not do any checking to assure the names are valid phases,
this is done at a later stage. Of additional interest are
\texttt{clearPhaseNames()} which deletes all current phase names, and
\texttt{getPhaseNames()} which returns an array of Strings with the phase names.

\texttt{timeTool.parsePhaseList("P,Pdiff,S,Sdiff,PKP,SKS");}

The next step is to correct the TauModel for the source depth.
The TauModel is created with a surface source, but can be corrected for
a source at depth, given in kilometers, with the \texttt{setSourceDepth(double)}
method. The actual depth correction is time consuming, and so is delayed until the calculate method is called. In addition,
if a correction was actually needed, it
calls \texttt{recalcPhases()} which verifies that the times and distances for
the phases in the phase list are compatible with the current model and depth.
\texttt{recalcPhases()} is also called
by \texttt{calculate()} in case changes were made to the list of phase names.

\texttt{timeTool.setSourceDepth(100.0);}

It remains only to calculate arrivals for a particular distance using the
\texttt{calculate(double)} method, which takes an angular distance in degrees.
The arrivals are stored as Arrival objects, which contain \texttt{time},
\texttt{dist}, \texttt{rayParam}, \texttt{sourceDepth}, and \texttt{name}
fields. The Arrivals can be accessed through either the
\texttt{getArrival(int)} method which returns the ith arrival, or
the \texttt{getArrivals()} method which returns an array of Arrivals.
Of additional interest is the \texttt{getNumArrivals()} method that returns
the number of arrivals.

\begin{verbatim}
timeTool.calculate(40);
Arrival[] arrivals = timeTool.getArrivals();
for (int i=0; i<arrivals.length; i++) {
   System.out.println(arrivals[i].getName+" arrives at "+
      (arrivals[i].getDist*180.0/Math.PI)+" degrees after "+
      arrivals[i].getTime+" seconds.");
}
\end{verbatim}

It is important to realize that all internal angular distances are
stored as radians, hence the conversion, and times in seconds.
This also means that the
ray parameters are stored as seconds per radian.



\subsection{Jacl}

One of the problems with Java based tools is that there is overhead associated
with starting a Java program due to the fact that the virtual machine must
first be started. While with normal interactive computing this is not such
a large problem, it can become very wastful when repeated calling a tool
from within a script. Significant time savings can be had if the tool
and its associated virtual machine can be kept alive for the duration
of the script.
Jacl, a Java implementation of the popular Tool Command Language or Tcl,
makes writing scripts that use the TauP Toolkit easy, and allows one
instance of both the
virtual machine as well as the tool to remain active for the whole script.
You may download jacl from http://www.scriptics.com/java.

Jacl allows a script to create Java objects, call any public method of
those objects and manipulate their attributes. Thus, creating a script
to do many similar calcuations or a custom application that makes these tools
usable in the way you want is as easy as writing a tcl script. We present a
brief walkthrough of a Jacl script that calculates pierce points for
numerous station event pairs.

The first three lines of the script should start up jacl. The second line is
a bit of trickery, it hides the third line from jacl while allowing sh to see
it. Jacl takes the backslash to be a line continuation marker, and therefore
accepts the third line as part of the comment started on the second line.
This just makes it easier to start up jacl without having to know the exact
path in advance.

\begin{verbatim}
#!/bin/sh
# \
exec jacl $0 $*
\end{verbatim}

Next, we will set up latitudes and longitudes for our stations and events.
This was modified from a script that read from a CSS database, but in
order to keep the script self contained, we have hardwired it here.
\begin{verbatim}
set slat(0) 35
set slon(0) -5
set elat(0) 125
set elon(0) 5
set edepth(0) 100
set elat(1) -10
set elon(1) 110
set edepth(1) 100
set elat(2) 40
set elon(2) 140
set edepth(2) 200
set elat(3) 65
set elon(3) -5
set edepth(3) 10
\end{verbatim}

Now we start up the pierce tool with the prem model
 and add the phases we are interested in. We will
only do P and S in PREM for simplicity.
\begin{verbatim}
set taup [java::new [list edu.sc.seis.TauP.TauP_Pierce String] "prem"]
$taup clearPhaseNames
$taup {parsePhaseList java.lang.String} "P,S"
\end{verbatim}

Here we get, and then loop over, all the discontinuities in the model in order
to find the one closest to 400 kilometers depth.
\begin{verbatim}
set disconArray [$taup getDisconDepths]
set maxDiff 99999999
set bestDepth 0
for {set i 0} {$i < [$disconArray length]} {incr i} {
   set depth [$disconArray get $i]
	if { [expr abs($depth - 400)] < $maxDiff} {
      set maxDiff [expr abs($depth - 400)]
      set bestDepth $depth
   }
}
\end{verbatim}

Loop over all events and stations and output the pierce point at the 400
kilometer discontinuity. We use the getLastPiercePoint(depth) method as
we want the receiver side pierce point. If we wanted the source side point
we could have used the getFirstPiercePoint(depth) method.
\begin{verbatim}
for {set eventIndex 0} {$eventIndex < [array size elat]} {incr eventIndex} {
   $taup depthCorrect $edepth($eventIndex)
   for {set staIndex 0} {$staIndex < [array size slat]} {incr staIndex} {
      set gcarc [java::call edu.sc.seis.TauP.SphericalCoords distance \
         $elat($eventIndex) $elon($eventIndex) \
         $slat($staIndex) $slon($staIndex)]
      set az     [java::call edu.sc.seis.TauP.SphericalCoords azimuth \
         $elat($eventIndex) $elon($eventIndex) \
         $slat($staIndex) $slon($staIndex)]
      $taup calculate $gcarc
      set numArrivals [$taup getNumArrivals]

      if {$numArrivals == 0} {
         puts "No arrivals for event $eventIndex"
      }
      for {set k 0} {$k< $numArrivals} {incr k} {
         set OneArrival [$taup getArrival $k]
         set name [ $OneArrival getName]

         if [ catch \
         {set OnePierce [$OneArrival getLastPiercePoint $bestDepth] }] {
            puts "$name doesn't pierce $bestDepth for event $eventIndex"
            continue
         }

         set dist [ $OnePierce getDist]
         set dist [expr $dist * (180./3.14159)]
         set plat [java::call edu.sc.seis.TauP.SphericalCoords latFor \
            $elat($eventIndex) $elon($eventIndex) $dist $az ]
         set plon [java::call edu.sc.seis.TauP.SphericalCoords lonFor \
            $elat($eventIndex) $elon($eventIndex) $dist $az ]
         puts [format "(%-7.3f, %-7.3f) $name from event number $eventIndex" \
            $plat $plon ]
      }
   }
}
\end{verbatim}

And here is the output:
\begin{verbatim}
piglet 56>./pierce.jacl
No arrivals for event 0
(-7.218 , 36.679 ) P from event number 1
(-7.214 , 36.676 ) S from event number 1
(-2.185 , 35.266 ) P from event number 2
(-2.205 , 35.264 ) S from event number 2
(-3.262 , 34.492 ) P from event number 3
(-3.142 , 34.457 ) S from event number 3
\end{verbatim}

This script, along with another simple travel time script, is included in the
distribution in the jacl subdirectory.


\subsection{C}

Here be monsters!

A C language interface to the TauP package is provided. A shared library
libtaup.so, provides access to the core functionality for generating travel
times. An example program using these interface routines is also provided,
gettimes.c.

A word of warning, nothing in Java is as frought with peril as JNI. This part of the TauP Toolkit is the hardest to get working. It is recommended that you be
fluent in both C and Java and have a high pain tolerance before attempting to
integrate TauP into a C program. In addition, the native code provided has not
been maintained and comes with no expectation that it will compile, much less work.
 You have been warned!


The native interface is distributed as C source code that you must compile
on your local machine. A makefile is provided to generate a shared library
and an example code to call the library. The makefile was created for use under
Solaris, but doesn't do anything particularly special, and should be
easily modifiable for other operating systems.

Of course, the system must be able to find this library, as well as the
Java libraries. Under Solaris, this can be accomplished with the
\texttt{LD\_LIBRARY\_PATH} environment variable. Other systems may vary.
The \texttt{CLASSPATH}  environment variable must also contain the taup.jar
file as well as the default java jar files. Note that under Java1.2 the command
line tools may work fine while the C interface has problems. This is due to
the java executable finding the standard files without the \texttt{CLASSPATH}.
The C interface bypasses the executable, and so does not benefit from this.
Properly setting the \texttt{CLASSPATH} is thus even more important for calling
Java from C.

The current C interface only provides method calls for the most basic
operations for getting travel times. If less common methods need to be called
then a quick look at the source code in the native directory
should be sufficient to create new hooks into those methods.

The state of the travel time calculator is preserved from call to call within
a TauPStruct structure. This contains references to the java virtual machine,
each of the method calls and the current model. This structure is always
the first argument to all of the method calls. While I believe this is the
least complicated style of interaction, it is not particularly memory
or processor efficient for uses involving more than one travel time calculator
active simultaneously. Primarily this is due to having more than one
java virtual machine running at the same time. Still, it is a
good example of how C can interact with Java.

The currently implemented method calls are:
\begin{center}
\begin{description}
\item[TauPInit] initializes the java virtual machine and properly fills in
the TauPStruct passed as the first argument. The second argument is the
name of the model to be used. The method signature is\newline
int TauPInit(TauPStruct *taupptr, char *modelName) {};

\item[TauPSetDepth] sets the source depth within the model. A initialized
TauPStruct is passed as the first argument, with the source depth
passed as the second. With the exception of creating a new model, this is
the most CPU intensive operation.
The method signature is\newline
int TauPSetDepth(TauPStruct taup, double depth) {};

\item[TauPClearPhases] clears any previously added phases. This should be
followed by a call to TauPAppendPhases, below, to add new phases.
An initialized TauPStruct is passed as the first argument.
The method signature is\newline
int TauPClearPhases(TauPStruct taup) {};

\item[TauPAppendPhases] appends new phases for calculation.
An initialized TauPStruct is passed as the first argument
and the phase names are passed as a comma or space separated string
in the second argument. All of the phase names that can be used in the
interactive code can be used here. Also, duplicates are checked for and
eliminated before being added. The method signature is\newline
int TauPAppendPhases(TauPStruct taup, char *phaseString) {};

\item[TauPCalculate] calculates all arrivals for all of the current
phases for the distance specified in the second argument.
An initialized TauPStruct is passed as the first argument.
The method signature is\newline
int TauPCalculate(TauPStruct taup, double degrees) {};

\item[TauPGetNumArrivals] returns the number of arrivals found with
the last call to TauPCalculate, above. A negative number indicates an
error. An initialized TauPStruct is passed as the first argument.
The method signature is\newline
int TauPGetNumArrivals(TauPStruct taup) {};

\item[TauPGetArrival] returns the ith arrival found with the last
call to TauPCalculate, above. The arrival is returned as a jobject, which
is mainly useful if it will be used as an argument for another java method
call. NULL is returned if an
error occurs. An initialized TauPStruct is passed as the first argument.
The method signature is\newline
jobject TauPGetArrival(TauPStruct taup, int arrivalNum) {};

\item[TauPGetArrivalName] returns the name of the ith arrival
found with the last call to TauPCalculate, above, as a
character pointer. An initialized TauPStruct is passed as the
first argument and the arrival number is passed as the second.
NULL is returned if there is an error.
The method signature is\newline
char * TauPGetArrivalName(TauPStruct taup, int arrivalNum) {};

\item[TauPGetArrivalPuristName] returns the purist's version of the
name of the ith arrival
found with the last call to TauPCalculate, above, as a
character pointer. The puris's name replaces depths with the true depth
of interfaces in the phase name, for example Pv410P might really be
Pv400P.
An initialized TauPStruct is passed as the
first argument and the arrival number is passed as the second.
NULL is returned if there is an error.
The method signature is\newline
char * TauPGetArrivalPuristName(TauPStruct taup, int arrivalNum) {};

\item[TauPGetArrivalTime] returns the travel time of the ith arrival
found with the last call to TauPCalculate, above.
An initialized TauPStruct is passed as the first argument and
the arrival number is passed as the second.
A negative number is returned if there is an error.
The method signature is\newline
double TauPGetArrivalTime(TauPStruct taup, int arrivalNum) {};

\item[TauPGetArrivalDist] returns the travel distance of the ith arrival
found with the last call to TauPCalculate, above.
An initialized TauPStruct is passed as the first argument and
the arrival number is passed as the second.
A negative number is returned if there is an error.
The method signature is\newline
double TauPGetArrivalDist(TauPStruct taup, int arrivalNum) {};

\item[TauPGetArrivalRayParam] returns the ray parameter of the ith arrival
found with the last call to TauPCalculate, above.
An initialized TauPStruct is passed as the first argument and
the arrival number is passed as the second.
A negative number is returned if there is an error.
The method signature is\newline
double TauPGetArrivalRayParam(TauPStruct taup, int arrivalNum) {};

\item[TauPGetArrivalSourceDepth] returns the source depth of the ith arrival
found with the last call to TauPCalculate, above.
An initialized TauPStruct is passed as the first argument and
the arrival number is passed as the second.
A negative number is returned if there is an error.
The method signature is\newline
double TauPGetArrivalSourceDepth(TauPStruct taup, int arrivalNum) {};

\item[TauPDestroy] destroys the java virtual machine and frees the used memory.
An initialized TauPStruct is passed as the first argument. A nonzero
error is returned if there is an error.
The method signature is\newline
int TauPDestroy(TauPStruct taup) {};

\end{description}
\end{center}
