
\section{Tools}

Tools included with the TauP package:

\begin{center}
\begin{tabular}{lp{3.2in}}

\texttt{taup\_time} & 
  calculates travel times. \\
\texttt{taup\_pierce} & 
  calculates pierce points at model discontinuities and specified depths. \\
\texttt{taup\_path} & calculates ray paths, depth versus epicentral distance. \\
\texttt{taup\_wavefront} & calculates wavefronts in steps of time, depth versus epicentral distance. \\
\texttt{taup} & a GUI that incorporates the time, pierce and path tools. This
requires swing, and hence may not work on some java1.1 systems. \\
\texttt{taup\_curve} & 
  calculates travel time curves, time versus epicentral distance. \\
\texttt{taup\_table} & outputs travel times for a range of depths and distances in an ASCII file \\
\texttt{taup\_setsac} & 
  puts theoretical arrival times into sac header variables. \\
\texttt{taup\_create} & 
  creates a .taup model from a velocity model. \\
\texttt{taup\_console} & Python scripting of TauP. \\
\end{tabular}
\end{center}

Each tool is a Java application and has an associated wrapper to make
execution easier: sh scripts 
for \textsc{Unix} and
bat files for windows.  The applications are machine independent but the 
wrappers are OS specific. 
For example, to invoke TauP\_Time under \textsc{Unix}, you could type

\texttt{java -Dtaup.model.path=\$\{TAUPPATH\} edu.sc.seis.TauP.TauP\_Time -mod prem}

or simply use the script that does the same thing,

\texttt{taup\_time -mod prem}

Each tool has a \texttt{--help} flag that will print a usage summary, as well
as a \texttt{--version} flag that will print the version.

\subsection{Default Parameters} \label{properties}

Each of the tools use Java Properties to allow the user to specify values 
for various
parameters. The properties all have default values, which are overridden by
values from a Properties file. The tools use \texttt{.taup} in the 
current directory, which overwrites values read in from 
\texttt{.taup} in the user's home directory. Properties may also be specified by
the \texttt{--prop} command line argument.
In addition, many of the properties can be overridden by command line arguments.

The form of the properties file is very simple. Each property is set using
the form 
\begin{verbatim}
taup.property.name=value
\end{verbatim}
 one property per line.
Comment lines are allowed, and begin with a \texttt{\#}.
Additionally, the names of all of the properties follow a convention of
prepending ``\texttt{taup.}'' to the name of the property. 
This helps to avoid name collisions when new properties
are added.

The currently used properties are:
\begin{description}

\item[taup.model.name] the name of the initial model to be loaded, 
iasp91 by default. 
\item[taup.model.path] search path for models. There is no default, 
but the value
in the .taup file will be concatinated with any value of taup.model.path 
from the system properties. For example, the environment variable TAUPPATH
is put into the system property taup.model.path by the wrapper shell scripts. 
\item[taup.source.depth] initial depth of the source, 0.0 km by default. 
\item[taup.phase.list] initial phase list, combined with taup.phase.file. The
defaults are p, s, P, S, Pn, Sn, PcP, ScS, Pdiff, Sdiff, PKP, SKS, PKiKP, 
SKiKS, PKIKP, SKIKS.  
\item[taup.phase.file] initial phase list, combined with taup.phase.list. There
is no default value, but the default value for taup.phase.list will not be 
used if there is a taup.phase.file property.
\item[taup.depth.precision] precision for depth output, the default is 1 decimal digit.
 Note that this is precision, not accuracy. Just 
because you get more digits doesn't imply that they have any meaning.
\item[taup.distance.precision] precision for distance output, 
the default is 2 decimal digits.
Note that this 
is precision, not accuracy. Just because you get more
digits doesn't imply that they have any meaning.
\item[taup.latlon.precision] precision for latitude and longitude output, the
default is 2 decimal digits. 
Note that this is precision, not accuracy. Just because you get more
digits doesn't imply that they have any meaning.
\item[taup.time.precision] precision for time, the default is 2 decimal digits.
Note that this is precision, not accuracy. Just because you get more
digits doesn't imply that they have any meaning.
\item[taup.rayparam.precision] precision for ray parameter, the default is 3 decimal digits.
Note that this is precision, not accuracy. Just because you get more
 digits doesn't imply that they have any meaning.
\item[taup.maxRefraction] The maximum degrees that a Pn or Sn can refract along the moho. Note this
 is not the total distance, only the segment along the moho. The default is 20 degrees.
\item[taup.maxDiffraction] The maximum degrees that a Pdiff or Sdiff can diffract along the CMB.
 Note this is not the total distance, only the segment along the CMB. The default is 60 degrees.
\item[taup.path.maxPathInc] maximum distance in degrees between points of a path. This does a simple linear interpolant between nearby values in order to make plots look better. There is noo improvement in the accuracy of the path.
\item[taup.table.locsat.maxdiff] maximum distance in degrees for which Pdiff 
or Sdiff are put into a locsat table. Beyond this distance Pdiff and Sdiff will 
not be added to the table, even though they may show up in the output of 
TauP\_Time. Instead, the next later arriving phase, if any, will be used 
instead. The default is 105 degrees.
\item[taup.create.minDeltaP] Minimum difference in slowness between 
successive slowness samples. This is used to decide when to stop adding new 
samples due to the distance check.
Used by TauP\_Create to create new models. 
The default is 0.1 sec/rad. 
\item[taup.create.maxDeltaP] Maximum difference in slowness between 
successive slowness samples. This is used to split any layers that exceed
this slowness gap.
Used by TauP\_Create to create new models. 
 The default is 11.0 sec/rad. 
\item[taup.create.maxDepthInterval] Maximum difference between successive depth 
samples. This is used immediately after reading in a velocity model, with
layers being split as needed.
Used by TauP\_Create to create new models. 
 The default is 115 km. 
\item[taup.create.maxRangeInterval] Maximum difference between successive 
ranges, in degrees. If the difference in distance for two adjacent rays
is greater than this, then a new slowness sample is inserted halfway between
the two existing slowness samples.
The default is 2.5 degrees.
\item[taup.create.maxInterpError] Maximum error for linear interpolation
 between successive sample in seconds. TauP\_Create uses this to try to insure
that the maximum error due to linear interpolation is less than this amount.
Of course, this is only an approximation based upon an estimate of the
 curvature of the travel time curve for surface focus turning waves. 
In particular, the error for more complicated phases is greater. For instance,
if the true error for P at 30 degrees is 0.03 seconds, then the error for
PP at 60 degrees would be twice that, 0.06 seconds.
Used by TauP\_Create to create new models. The default is 0.05 seconds.
\item[taup.create.allowInnerCoreS] Should we allow J phases, S in 
the inner core?
Used by TauP\_Create to create new models. 
 The default is true. Setting it to false slightly reduces storage and model 
load time.
\end{description}

Phase files, specified with the taup.phase.file property,
 are just text files with phase names, separated by either 
spaces, commas or newlines. In section \ref{phasenaming} the details of 
the phase naming convention are introduced. 
By and large, it is compatible with traditional 
seismological naming conventions, with a few additions and exceptions.
Also, for compatiblity with \textit{ttimes}, you may specify 
\texttt{ttp}, \texttt{ttp+}, \texttt{tts}, \texttt{tts+},
\texttt{ttbasic} or \texttt{ttall} to get a phase list corresponding 
to the \textit{ttimes} options.

\subsection{TauP\_Time}
TauP\_Time takes a either a {\texttt .taup} file generated by TauP\_Create or a velocity model in
{\texttt .nd} or {\texttt .tvel} format and generates 
travel times for specified phases through the given earth model. 

The usage is:
\begin{verbatim}
piglet 1>bin/taup_time --help
Usage: taup_time [arguments]
  or, for purists, java edu.sc.seis.TauP.TauP_Time [arguments]

Arguments are:
-ph phase list     -- comma separated phase list
-pf phasefile      -- file containing phases

-mod[el] modelname -- use velocity model "modelname" for calculations
                      Default is iasp91.

-h depth           -- source depth in km

--stadepth depth   -- receiver depth in km

Distance is given by:

-deg degrees       -- distance in degrees,
-km kilometers     -- distance in kilometers,
                      assumes radius of earth is 6371km,

or by giving the station and event latitude and lonitude,
                      assumes a spherical earth,

-sta[tion] lat lon -- sets the station latitude and longitude
-evt       lat lon -- sets the event latitude and longitude


--rayp             -- only output the ray parameter
--time             -- only output travel time
--rel phasename    -- also output relative travel time

--json             -- output travel times as json

-o [stdout|outfile]         -- output is redirected to stdout or to the "outfile" file
--prop [propfile]   -- set configuration properties
--debug             -- enable debugging output
--verbose           -- enable verbose output
--version           -- print the version
--help              -- print this out, but you already know that!
\end{verbatim} 

The modelname is from \texttt{modelname.taup}, a previously created file 
from TauP\_Create or a \texttt{modelname.nd} or \texttt{modelname.tvel} velocity model file. 
If there is insufficient information given on the command line, then you start
in interactive mode, otherwise it assumes you only want one set of times.

The phases are specified on the command line with the -ph option,
in a phase file with the -pf option, or in a properties file.
The model, phases, depth and distance can be changed within the interactive
section of TauP\_Time.

For example: \texttt{taup\_time -mod prem -h 200 -ph S,P -deg 57.4}

gives you arrival times for S and P for a 200 kilometer
deep source at a distance of 57.4 degrees.
\begin{verbatim}
piglet 2>taup_time -mod prem -h 200 -ph S,P -deg 57.4

Model: prem
Distance   Depth   Phase   Travel    Ray Param  Takeoff  Incident  Purist    Purist
  (deg)     (km)   Name    Time (s)  p (s/deg)   (deg)    (deg)   Distance   Name 
-----------------------------------------------------------------------------------
   57.40   200.0   P        566.77     6.968     31.18    21.31    57.40   = P    
   57.40   200.0   S       1028.59    13.017     32.34    22.00    57.40   = S    
\end{verbatim}

\subsection{TauP\_Pierce}

TauP\_Pierce uses a model like TauP\_Time but
determines the
angular distances from the epicenter at which the specified rays pierce 
discontinuities or specified depths in the model.

The usage is:
\begin{verbatim}
piglet 3>taup_pierce --help
Usage: taup_pierce [arguments]
  or, for purists, java edu.sc.seis.TauP.TauP_Pierce [arguments]

Arguments are:
-ph phase list     -- comma separated phase list
-pf phasefile      -- file containing phases

-mod[el] modelname -- use velocity model "modelname" for calculations
                      Default is iasp91.

-h depth           -- source depth in km

--stadepth depth   -- receiver depth in km

Distance is given by:

-deg degrees       -- distance in degrees,
-km kilometers     -- distance in kilometers,
                      assumes radius of earth is 6371km,

or by giving the station and event latitude and lonitude,
                      assumes a spherical earth,

-sta[tion] lat lon -- sets the station latitude and longitude
-evt       lat lon -- sets the event latitude and longitude


-az azimuth        -- sets the azimuth (event to station)
                      used to output lat and lon of pierce points
                      if the event lat lon and distance are also
                      given. Calculated if station and event
                      lat and lon are given.
-baz backazimuth   -- sets the back azimuth (station to event)
                      used to output lat and lon of pierce points
                      if the station lat lon and distance are also
                      given. Calculated if station and event
                      lat and lon are given.

-rev               -- only prints underside and bottom turn points, e.g. ^ and v
-turn              -- only prints bottom turning points, e.g. v
-under             -- only prints underside reflection points, e.g. ^

-pierce depth      -- adds depth for calculating pierce points
-nodiscon          -- only prints pierce points for the depths added with -pierce


-o [stdout|outfile]         -- output is redirected to stdout or to the "outfile" file
--prop [propfile]   -- set configuration properties
--debug             -- enable debugging output
--verbose           -- enable verbose output
--version           -- print the version
--help              -- print this out, but you already know that!
\end{verbatim} 
 
The \texttt{-rev}, \texttt{-turn} and \texttt{-under} flags are useful 
for limiting the output
to just those points you care about. The \texttt{-pierce depth} option 
allows you
to specify a ``pierce'' depth that does not correspond to an
actual discontinuity. For instance, where does a ray pierce 300 kilometers
above the CMB?

For example:

\texttt{taup\_pierce -mod prem -h 200 -ph S,P -deg 57.4}

would give you pierce points for S, and P for a 200 kilometer
deep source at a distance of 57.4 degrees. 

While

\texttt{taup\_pierce -turn -mod prem -h 200 -ph S,P -deg 57.4}

would give you just the points that each ray turns from downgoing to upgoing.

Using \texttt{-rev} would give you all points that the ray changes direction and \texttt{-under} gives just the underside reflections. 

Using the \texttt{-pierce} option

\texttt{taup\_pierce -mod prem -h 200 -ph S -sta 12 34.2 -evt -28 122 --pierce 2591 --nodiscon}

would give you just the points at which S crossed a depth of 2591 kilometers
from an event at ($28^\circ$ S, $122^\circ$ E)
to a station at ($12^\circ$ N, $34.2^\circ$ E). 
Because we specified the latitudes and longitudes, we also get the 
latitudes and longitudes of 
the pierce points, useful for making
a map view of where the rays encounter the chosen depth. Here is the output,
distance, depth, latitude and longitude, respectively.
\begin{verbatim}
> S at 1424.1 seconds at 93.7 degrees for a 200.0 km deep source in the prem model.
   31.58 2591.00    -17.86     89.39
   61.44 2591.00     -3.90     62.43
\end{verbatim}

\subsection{TauP\_Path}
TauP\_Path uses a model like TauP\_Time but
generates  the
angular distances from the epicenter at which the specified rays pierce 
path that the phases travel. The output is in GMT~\cite{GMT} ``psxy'' format, and is
placed into the file ``taup\_path.gmt''. 
If you specify the ``-gmt'' flag then this 
is a complete script with the appropriate ``psxy'' command prepended, so if you
have GMT installed, you can just:

\begin{verbatim}
taup_path -mod iasp91 -h 550 -deg 74 -ph S,ScS,sS,sScS -gmt
sh taup_path.gmt
ghostview taup_path.ps
\end{verbatim}

and you have a plot of the ray paths. To avoid possible plotting errors for
phases like \texttt{Sdiff}, the ray paths are interpolated to less than 
1 degree increments.

The usage is:
\begin{verbatim}
piglet 5>taup_path --help
Usage: taup_path [arguments]
  or, for purists, java edu.sc.seis.TauP.TauP_Path [arguments]

Arguments are:
-ph phase list     -- comma separated phase list
-pf phasefile      -- file containing phases

-mod[el] modelname -- use velocity model "modelname" for calculations
                      Default is iasp91.

-h depth           -- source depth in km

--stadepth depth   -- receiver depth in km

Distance is given by:

-deg degrees       -- distance in degrees,
-km kilometers     -- distance in kilometers,
                      assumes radius of earth is 6371km,

or by giving the station and event latitude and lonitude,
                      assumes a spherical earth,

-sta[tion] lat lon -- sets the station latitude and longitude
-evt       lat lon -- sets the event latitude and longitude


--gmt             -- outputs path as a complete GMT script.
--svg             -- outputs path as a complete SVG file.
--mapwidth        -- sets map width for GMT script.

-o [stdout|outfile]         -- output is redirected to stdout or to the "outfile" file
--prop [propfile]   -- set configuration properties
--debug             -- enable debugging output
--verbose           -- enable verbose output
--version           -- print the version
--help              -- print this out, but you already know that!
\end{verbatim} 

\subsection{TauP\_Wavefront}
TauP\_Wavefront is similar to TauP\_Path, but plots the wavefront at timesteps instead of the
ray paths. It also uses a model like TauP\_Time and
generates  depth versus
angular distances from the epicenter for the phases, but done at time slices instaed of depth slices. 
The output is in GMT~\cite{GMT} ``psxy'' format, and is
placed into the file ``taup\_wavefront.gmt''. 
The colums are distance, depth, time and ray param, although only the first two are used by the GMT script.
If you specify the ``-gmt'' flag then this 
is a complete script with the appropriate ``psxy'' command prepended, so if you
have GMT installed, you can just:

\begin{verbatim}
taup_wavefront -mod iasp91 -h 550 -ph S,ScS,sS,sScS --gmt
sh taup_wavefront.gmt
ghostview taup_wavefront.ps
\end{verbatim}

and you have a plot of the wavefronts.

The usage is:
\begin{verbatim}
piglet 5>taup_wavefront --help
Usage: taup_wavefront [arguments]
  or, for purists, java edu.sc.seis.TauP.TauP_Wavefront [arguments]

Arguments are:
-ph phase list     -- comma separated phase list
-pf phasefile      -- file containing phases

-mod[el] modelname -- use velocity model "modelname" for calculations
                      Default is iasp91.

-h depth           -- source depth in km

--stadepth depth   -- receiver depth in km

Distance is given by:

-deg degrees       -- distance in degrees,
-km kilometers     -- distance in kilometers,
                      assumes radius of earth is 6371km,

or by giving the station and event latitude and lonitude,
                      assumes a spherical earth,

-sta[tion] lat lon -- sets the station latitude and longitude
-evt       lat lon -- sets the event latitude and longitude


--rays  num      -- number of raypaths/distances to sample.
--timestep  num  -- steps in time (seconds) for output.
--timefiles      -- outputs each time into a separate .ps file within the gmt script.
--negdist        -- outputs negative distance as well so wavefronts are in both halves.

-o [stdout|outfile]         -- output is redirected to stdout or to the "outfile" file
--prop [propfile]   -- set configuration properties
--debug             -- enable debugging output
--verbose           -- enable verbose output
--version           -- print the version
--help              -- print this out, but you already know that!
\end{verbatim} 

\subsection{TauP}
TauP is unlike the rest of the tools in that it doesn't have any functionality
 beyond the other tools. It is just a GUI that uses TauP\_Time, TauP\_Pierce
and TauP\_Path. This is a nice feature of the java language in that each of 
these applications exists simultaneously as a library. The GUI does not 
currently have full access to all the things that these 
three tools can do, and certainly has a few rough edges, but can be useful
for browsing. Lastly, it currently does more work than it has to in that it
always calculates times, pierce points and paths, even if only one is actually
needed. So, it may be a bit pokey.

\subsection{TauP\_Curve}
TauP\_Curve creates a GMT style xy formated file of time versus distance. 
This can be used to create the familar travel time curves, but for
only the specified phases and depth. The curves are linearly interpolated 
between known sample points, and can thus be used to get a feel for the 
coarseness of sampling. For example, curves for s, S, ScS and Sdiff
for a 500 kilometer deep event in PREM could be generated by:

\texttt{taup\_curve -mod prem -h 500 -ph s,S,ScS,Sdiff --gmt}

The \texttt{-gmt} option prepends a GMT \texttt{psxy} command to the output 
file, creating a runnable script instead of just a data file. 
The output is put in taup\_curve.gmt by default, so to view the results:

\begin{verbatim}
sh taup_curve.gmt
ghostview taup_curve.ps
\end{verbatim}

The uasage is:
\begin{verbatim}
piglet 6>taup_curve --help
Usage: taup_curve [arguments]
  or, for purists, java edu.sc.seis.TauP.TauP_Curve [arguments]

Arguments are:
-ph phase list     -- comma separated phase list
-pf phasefile      -- file containing phases

-mod[el] modelname -- use velocity model "modelname" for calculations
                      Default is iasp91.

-h depth           -- source depth in km


--gmt              -- outputs curves as a complete GMT script.
-reddeg velocity   -- outputs curves with a reducing velocity (deg/sec).
-redkm velocity    -- outputs curves with a reducing velocity (km/sec).
-rel phasename     -- outputs relative travel time
--mapwidth width   -- sets map width for GMT script.

-o outfile         -- output is redirected to "outfile" instead of taup_curve.gmt
--prop [propfile]   -- set configuration properties
--debug            -- enable debugging output
--verbose          -- enable verbose output
--version          -- print the version
--help             -- print this out, but you already know that!
\end{verbatim} 
 
\subsection{TauP\_SetSac}
TauP\_SetSac uses the depth and distance information in 
\textsc{sac}~\cite{sacmanual}
 file headers to
put theoretical arrival times into the \texttt{t0}--\texttt{t9}
header variables. The header variable for a phase can be specified with by
a dash followed by a number, for instance \texttt{S-9} puts the S arrival time
in \texttt{t9}. If no header is specified then the time will be inserted in the
first header variable not allocated to another phase, starting with 0.
If there are no header variables not already allocated to a phase, then the
additional phases will not be added to the header. Note that this does not refer to times that are already in the \textsc{sac} file before TauP\_SetSac is run. They will be overwritten. The ray parameter, in seconds per radian, is also
inserted into the corresponding \texttt{user0}-\texttt{user9} header.

Note that triplicated phases are a problem as there is only one
spot to put a time. For example, in iasp91 S has three arrivals at 20~degrees but only
one can be put into the choosen header. TauP\_SetSac assumes that the first arrival
is the most important, and uses it. An improved method would allow a phase to have several
header variables associated with it, so that all arrivals could be marked. Currently
however, only the first arrival for a phase name is used.
 
\textbf{Warning:} TauP\_SetSac assumes the \textsc{evdp} header has depth in meters unless 
the -evdpkm
flag is used, in which case kilometers are assumed. This may be a problem for
users that improperly use kilometers for the depth units. Due to much
abuse of the \textsc{sac} depth header units, a warning message is
printed if the depth
appears to be in kilometers, i.e. it is $< 1000$, and -evdpkm is not used.
This can be safely ignored
if the event really is less than 1000 meters deep. See the \textsc{sac} 
manual~\cite{sacmanual} for confirmation.
 
The \textsc{sac} files must have \textsc{evdp} and the \textsc{o} marker set. 
Also, if \textsc{gcarc} or \textsc{dist} is not
set then TauP\_SetSac can calculate a distance only if 
\textsc{stla}, \textsc{stlo}, \textsc{evla} and \textsc{evlo}
are set.
 
The user should be very careful about previously set header variables. 
TauP\_SetSac will
overwrite any previously set \texttt{t} \texttt{user} headers. A future feature may do
more careful checking, but the current version makes no effort to verify that 
the header is undefined before writing.

If the given filename is a directory, TauP\_SetSac will recursively look for files within that directory to process. Thus,
a large directory structure of Sac files can be processed easily.

The usage is:
\begin{verbatim}
piglet 7>taup_setsac --help
Usage: taup_setsac [arguments]
  or, for purists, java edu.sc.seis.TauP.TauP_SetSac [arguments]

Arguments are:
-ph phase list     -- comma separated phase list,
                      use phase-# to specify the sac header,
                      for example, ScS-8 puts ScS in t8
-pf phasefile      -- file containing phases

--mod[el] modelname -- use velocity model "modelname" for calculations
                      Default is iasp91.


--evdpkm            -- sac depth header is in km, default is meters


--prop [propfile]   -- set configuration properties
--debug             -- enable debugging output
--verbose           -- enable verbose output
--version           -- print the version
--help              -- print this out, but you already know that!

sacfilename [sacfilename ...]

Ex: taup_setsac -mod S_prem -ph S-8,ScS-9 wmq.r wmq.t wmq.z
puts the first S arrival in T8 and ScS in T9
\end{verbatim} 

\subsection{TauP\_Table}

TauP\_Table creates an ASCII table of arrival times for a range of depths and
distances. Its main use is for generating travel time tables for earthquake
location programs such as LOCSAT. The \texttt{-generic} flag generates a flat
table with all arrivals at each depth and distance, one arrival per line.
The \texttt{-locsat} flag generates a LOCSAT style travel time table with
only the first arrival of all the phases listed at each distance and depth.
Thus, the program must be run several times in order to generate files for 
several phases. Also, both options write to standard out unless a file is
given with the -o flag.

There is a default phase, distance and depth list, but this is easily 
customizable with the \texttt{-header} option. An example LOCSAT style 
file for use as a header can be generated with 
\texttt{taup\_table -locsat -o example.locsat}. The first 
three sections specify the phase list, distances and depths to use. 
After editing, a custom table can be created with 
\texttt{taup\_table -header example.locsat}.

The usage is:
\begin{verbatim}
piglet 1>taup_table --help
Usage: taup_table [arguments]
  or, for purists, java edu.sc.seis.TauP.TauP_Table [arguments]

Arguments are:
-ph phase list     -- comma separated phase list
-pf phasefile      -- file containing phases

-mod[el] modelname -- use velocity model "modelname" for calculations
                      Default is iasp91.


-header filename   -- reads depth and distance spacing data
                      from a LOCSAT style file.
-generic           -- outputs a "generic" ascii table

-locsat            -- outputs a "locsat" style ascii table


-o [stdout|outfile]         -- output is redirected to stdout or to the "outfile" file
--prop [propfile]   -- set configuration properties
--debug             -- enable debugging output
--verbose           -- enable verbose output
--version           -- print the version
--help              -- print this out, but you already know that!
\end{verbatim}

\subsection{TauP\_Create}

TauP\_Create takes a ASCII velocity model file, samples the model
 and saves the tau model to a binary file. 
The output file holds all 
information about the model and need only be computed once. It
is used by all of the other tools. There are several parameters controlling
the density of sampling. Their values can be set with properties. See section
\ref{properties}, above.

Note that the use of TauP\_Create is no longer required as the various tools can read velocity models directly
and effectively call TauP\_Create internally. However, if a model file will be used repeatedly, using a
precomputed {\texttt .taup} file is more efficient.

The usage is:
\begin{verbatim}
piglet 8>taup_create --help
TauP_Create starting...
Usage: taup_create [arguments]
  or, for purists, java edu.sc.seis.TauP.TauP_Create [arguments]

Arguments are:

   To specify the velocity model:
-nd modelfile       -- "named discontinuities" velocity file
-tvel modelfile     -- ".tvel" velocity file, ala ttimes

--vplot file.gmt     -- plot velocity as a GMT script

--prop [propfile]    -- set configuration properties
--debug              -- enable debugging output
--verbose            -- enable verbose output
--version            -- print the version
--help               -- print this out, but you already know that!
\end{verbatim} 
 
\texttt{modelfile} is the ASCII text file holding the velocity model.
The \texttt{-nd} format is preferred 
because the depths, and thus identities, of the major internal boundaries can 
be unambiguously determined, making phase name parsing easier. 
See section \ref{exampleCreate} for an example.
For compatiblity, we support the \texttt{-tvel} format 
currently used by the latest ttimes package,~\shortcite{kennett:ak135}. 

The output will be a file named after the name of the 
velocity file, followed by \texttt{.taup}. For example

\texttt{taup\_create -nd prem.nd}

produces \texttt{prem.taup}.

\subsection{TauP\_Console}
TauP\_Console is an instance of the Jython, http://www.jython.org, interpreter with the TauP classes
preloaded along with some helper functionality. This allows python scripting with TauP.

Note, to make the distribution smaller, TauP\_Console is no longer compiled as the Jython jar was large. It is included in the source in the src/extras directory and also requires the jython jar.

The usage is:
\begin{verbatim}
taup_console [scriptfile]
\end{verbatim} 

If scriptfile is given, then that file is executed directly, if not then an interactive console is started.

An example of a simple console script is:
\begin{verbatim}
crotwell$ taup_console
taup> #
taup> # Set up some parameters
taup> stla = 13
taup> stlo = 14
taup> evla = -6
taup> evlo = -3
taup> evdp = 100
taup> dist = ellipDist(stla, stlo, evla, evlo)
taup> phNameList = ['P', 'S', 'p', 's']
taup> taup.sourceDepth = evdp
taup> tMod = loadTauModel('ak135')
taup> tModDepth = tMod.depthCorrect(evdp)
taup> phases = []
taup> for phName in phNameList:
...   phases.append(SeismicPhase(phName, tModDepth))
... 
taup> for phase in phases:
...   a = phase.getEarliestArrival(dist)
...   if a != None:
...     print "%s %3.2f"%(phase.name, a.time)
... 
P 317.37
S 576.18
taup> 
\end{verbatim}


