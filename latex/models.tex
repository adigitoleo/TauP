
\section{Creating and Saving Velocity Models}

\subsection{Velocity Model Files} \label{velfile}
There are currently two variations of velocity model files that can be read.
Both are piecewise linear between given depth points. Support for cubic spline
velocity models would be useful and is planned for a future release.

The first format is that used by the most recent ttimes
codes~\cite{kennett:ak135}, \texttt{.tvel}.
This format has two comment lines, followed by lines composed of depth, Vp, Vs and density, all separated by whitespace. TauP ignores the first two lines of this format and reads the remaining lines.

The second format is based on the format used by Xgbm,~\cite{xgbmreport,xgbmmanual}.
It is referred to here
as the \texttt{.nd} format for ``named discontinuities.''
Its biggest advantage is that it can specify the location of the major
boundaries and this makes it the preferred format.
The file consists of two types of lines, those that specify velocity at
a depth, and those that specify the name of a discontinuity.

The first type of line has between 3 and 6 numbers on a line separated
by whitespace. They are, in order, depth in kilometers to the sample point,
P velocity in kilometers per second,
S velocity in kilometers per second,
density in grams per cubic centimeter,
$Q_p$ attenuation for compressional waves and
$Q_s$ attenuation for shear waves.
Only  depth, $V_p$ and $V_s$ are required.
The remaining parameters, while not needed for travel time
calculations, are included to allow the model to be used for other purposes
in the future. The model is assumed to be linear between given depths and
repeated depths are used to represent discontinuities.

The second type of line within the \texttt{.nd} format specifies one of the
three major internal boundaries,
\textit{mantle} for the crust-mantle boundary,
\textit{outer-core} for the outer core-mantle boundary,
or \textit{inner-core} for the inner core-outer core boundary. These labels
are placed on a line by themselves between the two lines representing the
sample points above and below the depth of the
discontinuity.
These help to determine where a particular phase propagates. For instance,
in a model that has many crustal and upper mantle layers, from which
discontinuity does the phase \texttt{PvmP} reflect?
Explicit labeling eliminates potential ambiguity.

One further enhancement to these model file formats is the support for comments
embedded within the model files. As in shell scripting, everything after
a \# on a line is ignored. In addition, \textit{C} style \verb"/* ... */"
and \textit{C++} style \verb"// ..." comments are recognized.

A very simple \textit{named discontinuities} model file might look like this:
\begin{verbatim}/* below is a simple named discontinuities model. */
0.0  5.0  3.0  2.7
20   5.0  3.0  2.7
20   6.5  3.7  2.9
33   6.5  3.7  2.9
mantle          # the word "mantle" designates that this is the moho
33   7.8  4.4  3.3
410  8.9  4.7  3.5
410  9.1  4.9  3.7
670  10.2 5.5  4.0
670  10.7 5.9  4.4
2891 13.7 7.2  5.6
outer-core      # "outer-core" designates that this is the core mantle boundary
2891 8.0  0.0  9.9
5149.5 10.3 0.0 12.2
inner-core      # "inner-core" makes this the inner-outer core boundary
5149.5 11 3.5 12.7
6371 11.3  3.7  13 \end{verbatim}

In many cases it is better and easier to make use of taup velmerge
to create a new model by making changes to an existing global model, especially when for example the user only cares about crust and upper mantle structure and is happy with prem for
the lower mantle and core. Hand editing velocity model files
often results in hard to catch errors.

\subsection{Using Saved Tau Models}
There are three ways of finding a previously generated model file. First, as
a standard model as part of the distribution. Second, a list of directories and jar files to
be searched can be specified with the taup.model.path property.
Lastly, the path to the actual model file may be specified.
TauP searches each of these
places in order until it finds a model that matches the name.

\begin{enumerate}
\item Standard Model.

TauP first checks to see if the model name is associated with a standard model.
Several standard models are included within the distributed jar file.
They include
iasp91~\cite{iasp},
prem~\cite{dziewonski_anderson},
ak135~\cite{kennett:ak135},
jb~\cite{jb},
1066a~\cite{gilbert_dziewonski},
1066b~\cite{gilbert_dziewonski},
pwdk~\cite{weber_davis},
sp6~\cite{morelli} and
herrin~\cite{herrin}. Lastly, we have included qdt, which is a coarsely sampled version of iasp91~\cite{iasp}. It is samller, and thus loads quicker, but has significantly reduced accuracy.
We will consider adding other models to the distribution if
they are of wide interest.
They are included within the distribution jar file but
taup can locate them with just the model name.

\item Within the taup.model.path property.

Users can create custom models, and place the stored models in a convenient
location. If the taup.model.path property includes those
directories or jar files, then they can be located.
The search is done in the order of taup.model.path until a model matching the model
name is found. While taup.model.path is a Java property, the shell scripts provided
translate the environment variable TAUPPATH into this property. The user
generally need not be aware of this fact except when the tools are invoked
without using the provided shell scripts. A more desirable method is to
set the taup.model.path in a properties file. See section \ref{properties} for
more details.

The taup.model.path property is constructed in the manner of standard Java CLASSPATH
which is itself based loosely on the manner of the \textsc{Unix} PATH.
The only real
differences between CLASSPATH and PATH are that a jar file
may be placed directly in the path and the path separator character
is machine dependent, \textsc{Unix} is `:' but other systems may vary.

The taup.model.path allows you to have directories containing saved model files
as well as jar files of models.
For instance, in a \textsc{Unix} system using the c shell,
you could set your TAUPPATH to be, (all one line):

\begin{verbatim}
setenv TAUPPATH /home/xxx/MyModels.jar:/home/xxx/ModelDir:
/usr/local/lib/localModels.jar
\end{verbatim}

or you could place a line in the \texttt{.taup} file in your home directory
that accomplished the same thing, again all one line:

\begin{verbatim}
taup.model.path=/home/xxx/MyModels.jar:/home/xxx/ModelDir:
/usr/local/lib/localModels.jar
\end{verbatim}

If you place models in a jar, TauP assumes that they are placed
in a directory called \texttt{Models} before they are jarred.
For example, you might
use taup\_create to create several taup models in the Models directory
and then create a jar file.

\texttt{jar -cf MyModels.jar Models}

Including a ``.'' for the current working directory with the taup.model.path
is not necessary since we
always check there, see \ref{cwdmodel} below, but it may be used to
change the search order.

\item \label{cwdmodel} The last place TauP looks is for a model file specified
on the command line.
So, if you generate newModel.taup and want to get some times, you can just say:
\texttt{taup\_time -mod newModel.taup}
or even just
\texttt{taup\_time -mod newModel}
as TauP can add the taup suffix if necessary. A relative or absolute pathname
may precede the model, e.g.
\texttt{taup\_time -mod ../OtherDir/newModel.taup}.
New in version 2.0 is the ability of the tools to load a velocity model directly
and handle the create functionality internall, so in addition to .taup files,
the .nd and .tvel model files can be loaded if there is not a .taup file found.
Note that there is extra work involved in processing the velocity file, and so
frequently used models should still be converted using TauP\_Create to avoid
reprocessing them each time the tool starts.

\end{enumerate}
