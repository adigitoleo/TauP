
\section{Installing}
\label{install}

The installation for TauP under \textsc{Unix} is quite simple. And with Java's
platform independence, the package should be usable on a Mac or Windows
machine.

\subsubsection{Max OSX Install}

For Mac OSX, we recommend Homebrew. Installation is easy with:

\begin{verbatim}
  brew tap crotwell/crotwell
  brew install taup
  taup help
\end{verbatim}

\subsubsection{Manual Install}

\begin{enumerate}
\item Install a Java 1.8 or better virtual machine. If your system already has Java 1.8
or better installed then you can skip to the next step. You can test this
with ``java -version''. If it isn't there or the version is less than 1.8
you need to get and install Java. Most operating systems come with a version
of Java, and so this should not be an issue. If you do need to download Java,
point
your browser to https://openjdk.java.net/ and download and install
the Java SE Development Kit for Macintosh, Linux or Windows. TauP may run under earlier version of Java, but they are no longer supported.

\item Download TauP.X.X.X.tar.gz or TauP.X.X.X.zip. Make sure to get the most recent version, replacing the X's  in the file name. They can be found at

\texttt{http://www.seis.sc.edu/TauP}

\item Unpack the distribution.
\texttt{\newline gunzip TauP.X.X.X.tar.gz\newline tar -xvf TauP.X.X.X.tar\newline}
or
\texttt{\newline jar -xvf TauP.X.X.X.zip\newline or unzip TauP.X.X.X.zip}
This will create a directory called TauP-X.X.X. Inside
will be the files and directories listed in chapter 2, Distribution.

\item Put this directory someplace. It really doesn't matter where, although
a central place might make administration easier, /usr/local or
/usr/local/share are good choices. If you don't have superuser privileges
then your home directory is fine.

Previous versions of TauP recommended installing taup.jar as a \textit{standard extension} by placing the jar file into the jre/lib/ext subdirectory of your java installation. We no longer believe this is a good
practice, and if you have done this in the past you will need to remove the old version in order to
prevent a conflict.

\item Set the location of the TauP directory in your TAUP\_HOME environment variable.
This should be done in your .cshrc or .login. For instance, if you put Taup-2.6.0 in
/usr/local/share, then you could set the TAUP\_HOME to be:
\begin{verbatim}
setenv TAUP_HOME /usr/local/share/TauP-2.6.0
\end{verbatim}
for csh/tcsh or for bash
\begin{verbatim}
export TAUP_HOME=/usr/local/share/TauP-2.6.0
\end{verbatim}

Please make sure that an old version of taup.jar is not on your CLASSPATH as this will cause a
conflict.

\item Put the bin directory of the distribution directory in your PATH environment
variable, \begin{verbatim}${TAUP_HOME}/bin\end{verbatim} for instance. These wrapper
scripts are not essential, but they cut down on
typing. They are in the bin directory of the distribution
and are simple \textsc{Unix} sh scripts and \texttt{bat} files for windows.

\item Lastly, you may need to either source your .login and .cshrc files or
execute the \texttt{rehash} command to make the shell reevaluate the
contents of your PATH.

\end{enumerate}

That's it. If you have problems or encounter bugs, please mail them to me.
Please try to be as specific as possible. I am also interested in ideas for
additional features that might make this a more useful program.
Of course, I can make no promises,
but I would be glad to hear about them.

I can be reached via email at \textit{crotwell@seis.sc.edu}.
