

\section{Phase naming in TauP} \label{phasenaming}


A major feature of the TauP Toolkit is the implementation of a phase name parser
that allows the user to define essentially arbitrary phases through the earth.
Thus, the TauP Toolkit is extremely flexible in this respect since it is
not limited to a pre-defined set of phases.
Phase names are not hard-coded into the software, rather the names are interpreted
and the appropriate propagation path and resulting times are constructed at run time.
Designing a phase-naming convention that is general enough to support arbitrary phases
and easy to understand is an essential and somewhat challenging step.
The rules that we have developed are described here.
Most of phases resulting from these conventions should
be familiar to seismologists, e.g. pP, PP, PcS, PKiKP, etc.
However, the uniqueness required for parsing results in some new names for other
familiar phases.

In traditional ``whole-earth'' seismology, there are 3 major interfaces:  the free
surface, the core-mantle boundary, and the inner-outer core boundary.
Phases interacting with the core-mantle boundary and the inner core boundary are easy to
describe because the symbol for the wave type changes at the boundary (i.e. the symbol P
changes to K within the outer core even though the wave type is the same).
Phase multiples for these interfaces and the free surface are also easy to describe because
the symbols describe a unique path.
The challenge begins with the description of interactions with interfaces within the
crust and upper mantle.
We have introduced two new symbols to existing
nomenclature to provide unique descriptions of potential paths.
Phase names are constructed from a sequence of symbols and numbers (with no spaces)
that either describe the wave type, the interaction a wave makes with an interface, or
the depth to an interface involved in an interaction.

\begin{enumerate}
\item Symbols that describe wave-type are:

\begin{tabular}{lp{5.0in}}
\texttt{P} & compressional wave, upgoing or downgoing, in the crust or mantle \\
\texttt{p} & strictly upgoing P wave in the crust or mantle \\
\texttt{S} & shear wave, upgoing or downgoing, in the crust or mantle \\
\texttt{s} & strictly upgoing S wave in the crust or mantle \\
\texttt{K} & compressional wave in the outer core \\
\texttt{k} & strictly upgoing compressional wave in the outer core \\
\texttt{I} & compressional wave in the inner core \\
\texttt{J} & shear wave in the inner core \\
\end{tabular}
%\end{center}

\item Symbols that describe interactions with interfaces are:

%\begin{center}
\begin{tabular}{lp{5.0in}}
\texttt{m} & interaction with the moho \\
\texttt{g} & appended to P or S to represent a ray turning in the crust \\
\texttt{n} & appended to P or S to represent a head wave along the moho \\
\texttt{c} & topside reflection off the core mantle boundary \\
\texttt{i} & topside reflection off the inner core outer core boundary \\
\texttt{\^\,} & underside reflection, used primarily for crustal and mantle interfaces \\
\texttt{v} & topside reflection, used primarily for crustal and mantle interfaces \\
\texttt{V} & critical topside reflection, used primarily for crustal and mantle interfaces \\
\texttt{diff} & appended to P or S to represent a diffracted wave along the core mantle boundary \\
\texttt{kmps} & appended to a velocity to represent a horizontal phase velocity (see
\ref{kmps} below)\\
\texttt{ed} & appended to P or S to represent a exclusively downgoing path, for a receiver below the source (see
\ref{Ped} below)\\
\end{tabular}
%\end{center}

\item \label{Ped} The characters \texttt{p}, \texttt{s} and \texttt{k} \textbf{always} represent
up-going legs.
An example is the source to surface leg of the phase \texttt{pP}
from a source at depth.
\texttt{P} and \texttt{S} can be turning waves, but
always indicate downgoing waves leaving the source when they are the first symbol in a
phase name.
Thus, to get near-source, direct P-wave arrival times, you need to specify two
phases \texttt{p} and \texttt{P} or use the ``\textit{ttimes} compatibility phases'' described
below.
However, \texttt{P} may
represent a upgoing leg in certain cases.
For instance, \texttt{PcP} is
allowed since the direction of the phase is unambiguously determined by the symbol
\texttt{c}, but would be named \texttt{Pcp} by a purist using our nomenclature. The phase
\texttt{k} is similar to \texttt{p} but is an upgoing compressional wave in the outer core.

With the ability to have sources at depth, there is a need to specify the difference between a wave that is
exclusively downgoing to the receiver from one that turns and is upgoing at the receiver. The suffix \texttt{ed}
can be appended to indicate exclusively downgoing. So for a source at 10 km depth and a receiver at 20 km depth
at 0 degree distance the turning ray \texttt{P} does not have an arrival but \texttt{Ped} does.

\item Numbers, except velocities for \texttt{kmps}
phases (see \ref{kmps} below),
represent depths at which interactions take place.
For example, \texttt{P410s} represents a P-to-S conversion at a discontinuity at 410km
depth.
Since the S-leg is given by a lower-case symbol and no reflection indicator is
included, this represents a P-wave  converting to an S-wave when it hits the interface
from below.
The numbers given need not be the actual depth, the closest depth corresponding to a
discontinuity in the model will be used.
For example, if the time for \texttt{P410s} is requested in a model where the discontinuity
was really located at 406.7 kilometers depth, the time returned would actually be for
\texttt{P406.7s}.
The code ``taup time'' would note that this had been done via the ``Purist Name''.
Obviously, care should be taken to ensure that there are no other discontinuities
closer than the one of interest, but this approach allows generic interface
names like ``410'' and ``660'' to be used without knowing the exact depth in a given
model.

\item If a number appears between two phase legs, e.g. \texttt{S410P},
it represents a transmitted phase conversion, not a reflection.
Thus, \texttt{S410P} would be a transmitted conversion
from \texttt{S} to \texttt{P} at 410km depth.
Whether the conversion occurs
on the down-going side or up-going side is determined by the upper or lower
case of the following leg.
For instance, the phase \texttt{S410P}
propagates down as an \texttt{S}, converts at the 410
to a \texttt{P}, continues down, turns as a P-wave, and propagates back across the
410 and to the surface.
\texttt{S410p} on the other hand, propagates down
as a \texttt{S} through the 410, turns as an \texttt{S},
hits the 410 from the bottom, converts to a \texttt{p} and then goes up to the surface.
In these cases, the case of the phase symbol (P vs. p) is critical because the direction
of propagation (upgoing or downgoing) is not unambiguously defined elsewhere in the
phase name.
The importance is clear when you consider a source depth below 410 compared to above 410.
For a source depth greater than 410 km, \texttt{S410P} technically cannot exist while
\texttt{S410p} maintains the same path (a receiver side conversion) as it does for a
source depth above the 410.

The first letter can be lower case to indicate a conversion from
an up-going ray, e.g. \texttt{p410S} is a depth phase from
a source at greater than 410 kilometers depth that phase converts
at the 410 discontinuity.
It is strictly upgoing over
its entire path, and hence could also be labeled \texttt{p410s}.
\texttt{p410S} is often used to mean a reflection in the literature, but there
are too many possible interactions for the phase parser to allow this.
If the underside reflection is desired, use the \texttt{p\^\,410S} notation from
rule \ref{carrotv}.

\item Due to the two previous rules, \texttt{P410P} and \texttt{S410S}
are over specified, but still legal.
They are almost equivalent to \texttt{P} and \texttt{S}, respectively,
but restrict the path to phases transmitted through (turning below) the 410.
This notation is useful to
limit arrivals to just those that turn deeper than a discontinuity (thus avoiding
travel time curve triplications), even though they have no real interaction with it.

\item \label{carrotv}
The characters \texttt{\^\,}, \texttt{v} and \texttt{V} are new symbols introduced here to
represent bottom-side and top-side reflections, respectively.
They are followed by a number to
represent the approximate depth of the reflection or
a letter for standard discontinuities, \texttt{m}, \texttt{c} or \texttt{i}.
The lower-case \texttt{v} represents a generic reflection while \texttt{V} is
a critical reflection. Note however, that  \texttt{V} is critical in the sense of
without phase conversion. In other words, \texttt{PVmp} is critical for ray parameters
where a P wave cannot propagate into the mantle, regardless of whether
or not S can propagate. A critical reflection phase using \texttt{V} is always
a subset of the non-critical reflection using \texttt{v}.
Reflections from discontinuities besides the
core-mantle boundary, \texttt{c};
or inner-core outer-core boundary, \texttt{i}, must use the \texttt{\^\,}
and \texttt{v} notation.
For instance, in the TauP convention, \texttt{p\^\,410S} is used to describe
a near-source underside reflection.

Underside reflections, except at the
surface (\texttt{PP}, \texttt{sS}, etc.),
core-mantle boundary (\texttt{PKKP}, \texttt{SKKKS}, etc.), or
outer-core-inner-core boundary (\texttt{PKIIKP}, \texttt{SKJJKS},
\texttt{SKIIKS}, etc.), must
be specified with the \texttt{\^\,} notation.
For example, \texttt{P\^\,410P} and
\texttt{P\^\,mP} would both be underside
reflections from the 410km discontinuity and the Moho, respectively.
Because of the difficultly of creating interfaces where critical underside reflections
can occur in earth-like models, we have not added this capability.

The phase \texttt{PmP}, the traditional name for a top-side reflection from the Moho
discontinuity, must change names under our new convention.
The new name is \texttt{PvmP} or \texttt{Pvmp}
while \texttt{PmP} just describes a P-wave that turns beneath the Moho.
The reason the Moho must be handled differently from the core-mantle boundary is that
traditional nomenclature did not introduce a phase symbol change at the Moho.
Thus, while \texttt{PcP} makes sense since a P-wave in the core would be labeled
\texttt{K}, \texttt{PmP} could have several meanings.
The \texttt{m} symbol just allows the user to describe phases interaction with the Moho
without knowing its exact depth.
In all other respects, the \texttt{\^\,}-\texttt{v} nomenclature is maintained.

\item
Currently, \texttt{\^\,} and \texttt{v} for non-standard
discontinuities are allowed only in
the crust, mantle and outer core. Thus there are no reflections off non-standard
discontinuities within the inner core, primarily due to the lack of naming convention.
Reflections such as \texttt{PKiKP} and \texttt{PKIIKP} are
still fine.
There is no
reason in principle to restrict reflections off discontinuities in the
inner core, but until there is interest expressed, these phases will not be added.
Also, a naming convention would have to be created since
``\texttt{p}~is~to~\texttt{P}'' is not the same as
``\texttt{i}~is~to~\texttt{I}''.

\item Currently there is no support for \texttt{PKPab}, \texttt{PKPbc},
or \texttt{PKPdf} phase names.
They lead to increased algorithmic complexity that at this point seems
unwarranted, and TauP uses phase names to describe paths, but
\texttt{PKPab} and \texttt{PKPbc} differentiate between two arrivals from
the same path, \texttt{PKP}.
Currently, in regions where triplications develop, the triplicated phase will have multiple
arrivals at a given distance.
So, \texttt{PKPab} and \texttt{PKPbc} are
both labeled just \texttt{PKP} while \texttt{PKPdf} is called \texttt{PKIKP}.

\item \label{kmps}
The symbol \texttt{kmps} is used to get the travel time for a
specific horizontal phase velocity.
For example, \texttt{2kmps} represents a horizontal phase
velocity of 2 kilometers per second.
While the calculations for these are trivial, it is convenient
to have them available to estimate surface wave travel times or to define windows of
interest for given paths.

\item As a convenience, a \textit{ttimes} phase name compatibility mode is available.
So \texttt{ttp} gives
you the phase list corresponding to \texttt{P} in \textit{ttimes}.
Similarly there are \texttt{tts}, \texttt{ttp+},
\texttt{tts+}, \texttt{ttbasic} and \texttt{ttall}.

\end{enumerate}
