
\section{Distribution}


\subsection{What and Where}
The current distribution of the TauP package is 1.1, dated February 9, 2001.

The distribution directory obtained from either the gzipped tar file or the jar file contains:

\begin{center}
\begin{tabular}{lp{4in}}
README & getting started information \\
taup.jar & the jar file with all the classes and standard models \\
taup.html & a simple web page that loads a rudimentary applet to use the TauP
package. Be warned that this is not really meant to be used over the Internet
as the download time for 1.5Mb may be too great and most browsers do not yet
support the 1.1 version of Java. \\
exampleProperties & example properties file \\
HISTORY & change log \\
License & the free for non-commercial license \\
bin & a directory with wrapper scripts appropriate for 
\textsc{Unix} installations\\
doc & a directory with Postscript and pdf versions of this manual. \\
html & a directory with the javadoc output from
the source code, mainly useful
for writing new java programs that use the TauP package. \\
jacl & a directory with Jacl examples for accessing the TauP package directly within scripts. \\
native & a directory with a C library and example program that use the
Java Native Interface, providing a basic interface between C programs
and the TauP package. \\
Maple & a directory with Maple scripts showing the time and distance equations
used. \\
src & a directory with all of the java source code. \\
modelFiles & a directory with the text files for each of the velocity models.
\end{tabular}
\end{center}

The taup.jar file contains everything needed for a working version of the package.
This greatly simplifies the installation process and reduces potential errors.
See appendix \ref{install} for detailed installation instructions.

\subsection{Pros and Cons of the Current Release}
The increased flexibility of this package provides significant advantages. Among
these are:
\begin{enumerate}
\item The ability to use many different models. We include a variety of previously created
models as well as the option of creating your own models. A conscious effort 
was made to make as few assumptions as possible about the nature of the model. 
Therefore,
even models that have very different structures than common global models can be
used.

\item Phase parsing. Phases are not hard coded into the program, instead the phase
names are parsed. This creates an opportunity for the study of less common
phases that are not present in previous travel time calculators. 

\item Programming interface for Java. Because of the use of the Java programming
language, all of the tools exist simultaneously as both applications and libraries.
Thus, any Java code that has a need for travel times can load and manipulate
the objects within this package. In addition, Jacl, the Java implementation of the popular Tcl scripting language, provides a simple means of directly accessing the public methods within the package.

\end{enumerate}

Of course, there are always drawbacks. The main difficulty at present is speed. 
The tools in this package are not as fast as natively compiled \textit{C} or
\textsc{Fortran}. Execution speed, however, is not always the best measure 
of usefulness. A extremely fast code that can't use your velocity model, or that
won't run on your machine is worse than a slower, but more flexible tool. In
addition, processor speed is increasing at a fast rate, and codes 
that were considered
too slow yesterday, are usable today, and will have insignificant execution times
tomorrow. One last point is that there is a significant effort within the 
commercial world to improve the speed of Java. 
The educational and research
world will benefit significantly from these efforts without incurring any cost.

 
\subsection{Future Plans}

There are several ideas for improvements that we may pursue, such as:

\begin{enumerate}
\item A GUI. A graphical user interface would greatly improve the usefulness
of this package, especially for non command line uses such as on the Macintosh
or within web browsers. The beginings of such a GUI are there in the TauP tool,
but at present it cannot access all of the functionality of the tools.

\item Non-\textsc{Unix} platforms. In spite of Java's platform neutral nature,
our installation instructions and reliance on command line style input and output limits the usability of the package on non-\textsc{Unix} operating systems. 
We would like to extend support as much as possible to other operating
systems. This should not be a large effort, and thus should be completed soon.

\item Use of the $\tau$ function. In spite of the name, TauP does not yet use
Tau splines. At present I do not believe that this would provide a large
improvement over the current linear interpolation, but it is likely worth doing.

\item Web based applet. One of Java's main uses currently is for the development of web based applets. An applet is a small application that is downloaded and
executed within a web browser. This is an attractive opportunity and we have a simple
example of one included in this distribution. 
There are difficulties as the network time to download the 
model files may be unacceptable, as well as the lack of support for Java~1.1 in current browsers. A client server architecture as well as the continued improvement of commercial web browsers
may be able to address these issues.

\item 1.1D models. There is nothing in the method that requires the source and
receiver velocity models to be the same. With this idea, a separate crustal 
model appropriate to each region could be used for the source and receiver.

\item WKBJ synthetics. The calculation of $\tau$ is a necessary step for WKBJ 
synthetics, and so this is a natural direction. It likely involves significant 
effort, however.
\end{enumerate}
