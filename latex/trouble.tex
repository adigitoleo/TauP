
\section{Troubleshooting}

There are a few idiosyncrocies about the codes and Java in general that you may
run into.

\begin{enumerate}

\item Out of memory errors. Java's default maximum memory for most uses should be fine,
 but some very complicated
models using a large number of phases may exceed this limit. A simple fix is
to change the maximum memory to be a larger amount. The -Xmx command line argument
to the java command does this. So, to set the maximum amount of memory to
128 megabytes you could say \texttt{java -Xmx128m edu.sc.seis.TauP.TauP\_Path}.
For convenience you may wish to make this change more permanent by adding it
to the scripts, i.e. taup\_time, etc. Also note that the m at the end specifies megabytes, if you 
omit it, then the value is 128 bytes, which is probably not what you want.

\item Garbled jar files. Care should be taken with the jar files when
transferring them
from one operating system to another. Certain file transfer utilities
make an attempt
to \textit{fix} text files by changing \texttt{RETURN LINEFEED} sequences to
just \texttt{LINEFEED} or just \texttt{RETURN} or vice versa. This is useful for
real text files, but dangerous for jar files. I have noticed this when
transferring
files between \textsc{Unix} and Macintosh, and it likely can happen
between any two
operating systems with differing end of line identifiers. Using binary mode for
ftp transactions is likely wise.

\item Trouble with bat files. I don't use windows and so I do not know if the
bat files are really useful or not. If you find a better method, I would be
happy to include it.
\end{enumerate}
