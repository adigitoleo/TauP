 
\section{Troubleshooting}

There are a few idiosyncrocies about the codes and Java in general that you may
run into.

\begin{enumerate}

\item Out of memory errors. By default Java sets its maximum memory to be
16 megabytes. For most uses this is sufficient, but some very complicated
models using a large number of phases may exceed this limit. A simple fix is
to change the maximum memory to be a larger amount. The -mx command line argument
to the java command does this. So, to set the maximum amount of memory to
32 megabytes you could say \texttt{java -mx32m edu.sc.seis.TauP.TauP\_Path}.
For convenience you may wish to make this change more permanent by adding it
to the scripts, i.e. taup\_time, etc.

\item Garbled jar files. Care should be taken with the jar files when 
transferring them
from one operating system to another. Certain file transfer utilities 
make an attempt
to \textit{fix} text files by changing \texttt{RETURN LINEFEED} sequences to
just \texttt{LINEFEED} or just \texttt{RETURN} or vice versa. This is useful for
real text files, but dangerous for jar files. I have noticed this when 
transferring
files between \textsc{Unix} and Macintosh, and it likely can happen 
between any two
operating systems with differing end of line identifiers. Using binary mode for
ftp transactions is likely wise.

\item Trouble with applet and appletviewer. There have been problems with 
appletviewer
and the applet related to the CLASSPATH. For the applet to work, 
taup.jar needs to
be in the CLASSPATH. However, other files in the CLASSPATH may confuse either
the applet when it looks for models, or cause the appletviewer to not run at all
with any applets. If you experience any problems with TauPApplet or 
appletviewer in
general, try running with a simplified CLASSPATH.  For example, on
\textsc{Unix} machines in the csh:
\begin{verbatim}
setenv CLASSPATH /path/to/the/jar/taup.jar
appletviewer taup.html
\end{verbatim}

\item Trouble with bat files. I don't use windows and so I do not know if the
bat files are really useful or not. If you find a better method, I would be 
happy to include it.
\end{enumerate}

